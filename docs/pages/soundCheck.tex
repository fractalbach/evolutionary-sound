\documentclass[]{article}
\usepackage{lmodern}
\usepackage[hyphenbreaks]{breakurl}
\usepackage[hyphens]{url}
\usepackage{fullpage}
\usepackage{amssymb,amsmath}
\usepackage{ifxetex,ifluatex}
\usepackage{fixltx2e} % provides \textsubscript
\ifnum 0\ifxetex 1\fi\ifluatex 1\fi=0 % if pdftex
  \usepackage[T1]{fontenc}
  \usepackage[utf8]{inputenc}
\else % if luatex or xelatex
  \ifxetex
    \usepackage{mathspec}
  \else
    \usepackage{fontspec}
  \fi
  \defaultfontfeatures{Ligatures=TeX,Scale=MatchLowercase}
\fi
% use upquote if available, for straight quotes in verbatim environments
\IfFileExists{upquote.sty}{\usepackage{upquote}}{}
% use microtype if available
\IfFileExists{microtype.sty}{%
\usepackage{microtype}
\UseMicrotypeSet[protrusion]{basicmath} % disable protrusion for tt fonts
}{}
\usepackage[unicode=true]{hyperref}
\hypersetup{
            pdfborder={0 0 0},
            breaklinks=true}
\urlstyle{same}  % don't use monospace font for urls
\usepackage{color}
\usepackage{fancyvrb}
\newcommand{\VerbBar}{|}
\newcommand{\VERB}{\Verb[commandchars=\\\{\}]}
\DefineVerbatimEnvironment{Highlighting}{Verbatim}{commandchars=\\\{\}}
% Add ',fontsize=\small' for more characters per line
\newenvironment{Shaded}{}{}
\newcommand{\KeywordTok}[1]{\textcolor[rgb]{0.00,0.44,0.13}{\textbf{#1}}}
\newcommand{\DataTypeTok}[1]{\textcolor[rgb]{0.56,0.13,0.00}{#1}}
\newcommand{\DecValTok}[1]{\textcolor[rgb]{0.25,0.63,0.44}{#1}}
\newcommand{\BaseNTok}[1]{\textcolor[rgb]{0.25,0.63,0.44}{#1}}
\newcommand{\FloatTok}[1]{\textcolor[rgb]{0.25,0.63,0.44}{#1}}
\newcommand{\ConstantTok}[1]{\textcolor[rgb]{0.53,0.00,0.00}{#1}}
\newcommand{\CharTok}[1]{\textcolor[rgb]{0.25,0.44,0.63}{#1}}
\newcommand{\SpecialCharTok}[1]{\textcolor[rgb]{0.25,0.44,0.63}{#1}}
\newcommand{\StringTok}[1]{\textcolor[rgb]{0.25,0.44,0.63}{#1}}
\newcommand{\VerbatimStringTok}[1]{\textcolor[rgb]{0.25,0.44,0.63}{#1}}
\newcommand{\SpecialStringTok}[1]{\textcolor[rgb]{0.73,0.40,0.53}{#1}}
\newcommand{\ImportTok}[1]{#1}
\newcommand{\CommentTok}[1]{\textcolor[rgb]{0.38,0.63,0.69}{\textit{#1}}}
\newcommand{\DocumentationTok}[1]{\textcolor[rgb]{0.73,0.13,0.13}{\textit{#1}}}
\newcommand{\AnnotationTok}[1]{\textcolor[rgb]{0.38,0.63,0.69}{\textbf{\textit{#1}}}}
\newcommand{\CommentVarTok}[1]{\textcolor[rgb]{0.38,0.63,0.69}{\textbf{\textit{#1}}}}
\newcommand{\OtherTok}[1]{\textcolor[rgb]{0.00,0.44,0.13}{#1}}
\newcommand{\FunctionTok}[1]{\textcolor[rgb]{0.02,0.16,0.49}{#1}}
\newcommand{\VariableTok}[1]{\textcolor[rgb]{0.10,0.09,0.49}{#1}}
\newcommand{\ControlFlowTok}[1]{\textcolor[rgb]{0.00,0.44,0.13}{\textbf{#1}}}
\newcommand{\OperatorTok}[1]{\textcolor[rgb]{0.40,0.40,0.40}{#1}}
\newcommand{\BuiltInTok}[1]{#1}
\newcommand{\ExtensionTok}[1]{#1}
\newcommand{\PreprocessorTok}[1]{\textcolor[rgb]{0.74,0.48,0.00}{#1}}
\newcommand{\AttributeTok}[1]{\textcolor[rgb]{0.49,0.56,0.16}{#1}}
\newcommand{\RegionMarkerTok}[1]{#1}
\newcommand{\InformationTok}[1]{\textcolor[rgb]{0.38,0.63,0.69}{\textbf{\textit{#1}}}}
\newcommand{\WarningTok}[1]{\textcolor[rgb]{0.38,0.63,0.69}{\textbf{\textit{#1}}}}
\newcommand{\AlertTok}[1]{\textcolor[rgb]{1.00,0.00,0.00}{\textbf{#1}}}
\newcommand{\ErrorTok}[1]{\textcolor[rgb]{1.00,0.00,0.00}{\textbf{#1}}}
\newcommand{\NormalTok}[1]{#1}
\IfFileExists{parskip.sty}{%
\usepackage{parskip}
}{% else
\setlength{\parindent}{0pt}
\setlength{\parskip}{6pt plus 2pt minus 1pt}
}
\setlength{\emergencystretch}{3em}  % prevent overfull lines
\providecommand{\tightlist}{%
  \setlength{\itemsep}{0pt}\setlength{\parskip}{0pt}}
\setcounter{secnumdepth}{0}
% Redefines (sub)paragraphs to behave more like sections
\ifx\paragraph\undefined\else
\let\oldparagraph\paragraph
\renewcommand{\paragraph}[1]{\oldparagraph{#1}\mbox{}}
\fi
\ifx\subparagraph\undefined\else
\let\oldsubparagraph\subparagraph
\renewcommand{\subparagraph}[1]{\oldsubparagraph{#1}\mbox{}}
\fi

% set default figure placement to htbp
\makeatletter
\def\fps@figure{htbp}
\makeatother


\date{}

\begin{document}


\section{Sound Check Experiment}\label{sound-check-experiment}

This is a preliminary test for making sound with javascript. Right now,
the code will be tangled together in an attempt to understand how
everything works. The actual program will have much more separation of
parts.

\subsection{Global Constants}\label{global-constants}

The Lists will be constant throughout the whole program. The program
will randomly select an item from one of these lists depending on the
situation, and use that value to affect the sound in some way.

\begin{Shaded}
\begin{Highlighting}[]
\KeywordTok{const}\NormalTok{ LIST_OF_WAVEFORM_TYPES }\OperatorTok{=}\NormalTok{ [}
    \StringTok{'sine'}\OperatorTok{,} \StringTok{'square'}\OperatorTok{,} \StringTok{'sawtooth'}\OperatorTok{,} \StringTok{'triangle'}
\NormalTok{]}\OperatorTok{;}
\end{Highlighting}
\end{Shaded}

\subsection{Global Functions}\label{global-functions}

During the creation of random sounds, randomly generating numbers is a
common occurence. This function generates a random integer between two
integers.

More precisely, let \(f(m,M)\) be a function
\(f:(\mathrm{Z}, \mathrm{Z}) \to \mathrm{Z})\) such that \(f(m,M) = x\)
and \(m \leq x \leq M\)

\begin{Shaded}
\begin{Highlighting}[]
\KeywordTok{function} \AttributeTok{getRandomInt}\NormalTok{(min}\OperatorTok{,}\NormalTok{ max) }\OperatorTok{\{}
\NormalTok{    min }\OperatorTok{=} \VariableTok{Math}\NormalTok{.}\AttributeTok{ceil}\NormalTok{(min)}\OperatorTok{;}
\NormalTok{    max }\OperatorTok{=} \VariableTok{Math}\NormalTok{.}\AttributeTok{floor}\NormalTok{(max) }\OperatorTok{+} \DecValTok{1}\OperatorTok{;}
    \ControlFlowTok{return} \VariableTok{Math}\NormalTok{.}\AttributeTok{floor}\NormalTok{(}\VariableTok{Math}\NormalTok{.}\AttributeTok{random}\NormalTok{() }\OperatorTok{*}\NormalTok{ (max }\OperatorTok{-}\NormalTok{ min)) }\OperatorTok{+}\NormalTok{ min}\OperatorTok{;}
\OperatorTok{\}}
\end{Highlighting}
\end{Shaded}

Since the most common target is a random value from one of the global
Constant Arrays, a useful function will return one of those randomly
selected values. This can be achieved by picking a random KEY in the
array, with the minimum being 0, and the maximum being the length of the
array.

\begin{Shaded}
\begin{Highlighting}[]
\KeywordTok{function} \AttributeTok{getRandomValueFromArray}\NormalTok{(name) }\OperatorTok{\{}
    \ControlFlowTok{return}\NormalTok{ name[ }\AttributeTok{getRandomInt}\NormalTok{(}\DecValTok{0}\OperatorTok{,} \VariableTok{name}\NormalTok{.}\AttributeTok{length} \OperatorTok{-} \DecValTok{1}\NormalTok{) ]}\OperatorTok{;}
\OperatorTok{\}}
\end{Highlighting}
\end{Shaded}

\subsection{Audio Context}\label{audio-context}

Quote from
\href{https://developer.mozilla.org/en-US/docs/Web/API/AudioContext}{Mozilla
Developer - Audio Context}

\begin{quote}
The AudioContext interface represents an audio-processing graph built
from audio modules linked together, each represented by an AudioNode. An
audio context controls both the creation of the nodes it contains and
the execution of the audio processing, or decoding. You need to create
an AudioContext before you do anything else, as everything happens
inside a context.
\end{quote}

\begin{Shaded}
\begin{Highlighting}[]
\KeywordTok{var}\NormalTok{ audioContext }\OperatorTok{=} \KeywordTok{new}\NormalTok{ (}\VariableTok{window}\NormalTok{.}\AttributeTok{AudioContext} \OperatorTok{||} \VariableTok{window}\NormalTok{.}\AttributeTok{webkitAudioContext}\NormalTok{)()}\OperatorTok{;}
\end{Highlighting}
\end{Shaded}

\subsection{Simple Oscillator}\label{simple-oscillator}

For this example, an oscillator node will be created. Given that we have
a list of waveforms, we will choose one of them at random upon creation
of the oscillator.

\begin{Shaded}
\begin{Highlighting}[]
\KeywordTok{var}\NormalTok{ buildRandomOscillator }\OperatorTok{=} \KeywordTok{function}\NormalTok{ () }\OperatorTok{\{}
    \KeywordTok{var}\NormalTok{ osc}\OperatorTok{;}
    \KeywordTok{var}\NormalTok{ waveformType }\OperatorTok{=} \AttributeTok{getRandomValueFromArray}\NormalTok{(LIST_OF_WAVEFORM_TYPES)}\OperatorTok{;}
\NormalTok{    osc }\OperatorTok{=} \VariableTok{audioContext}\NormalTok{.}\AttributeTok{createOscillator}\NormalTok{()}\OperatorTok{;}
    \VariableTok{osc}\NormalTok{.}\AttributeTok{type} \OperatorTok{=}\NormalTok{ waveformType}\OperatorTok{;}
    \ControlFlowTok{return}\NormalTok{ osc}\OperatorTok{;}
\OperatorTok{\};}  
\end{Highlighting}
\end{Shaded}

\subsection{Button Events - Start and Stop the
Sound}\label{button-events---start-and-stop-the-sound}

For now, we want an oscillator to be defined globally, so that we can
turn it on or off. Before any buttons are pressed, there is no sound. We
just have an empty variable where the oscillator will go when it is
turned on.

\begin{Shaded}
\begin{Highlighting}[]
\KeywordTok{var}\NormalTok{ osc}\OperatorTok{;}
\KeywordTok{var}\NormalTok{ is_sound_on }\OperatorTok{=} \KeywordTok{false}\OperatorTok{;}
\end{Highlighting}
\end{Shaded}

These functions simply start and stop the sound, and are intended to be
used by button events on the HTML page. This is part of the interface,
and should always be present in some form, just in case somebody wants
to shut off the sound.

\begin{Shaded}
\begin{Highlighting}[]
\KeywordTok{var}\NormalTok{ startSound }\OperatorTok{=} \KeywordTok{function}\NormalTok{() }\OperatorTok{\{}
    \ControlFlowTok{if}\NormalTok{ (is_sound_on) }\OperatorTok{\{}\ControlFlowTok{return}\OperatorTok{;\}}
\NormalTok{    osc }\OperatorTok{=} \AttributeTok{buildRandomOscillator}\NormalTok{()}\OperatorTok{;}
    \VariableTok{osc}\NormalTok{.}\AttributeTok{connect}\NormalTok{(}\VariableTok{audioContext}\NormalTok{.}\AttributeTok{destination}\NormalTok{)}\OperatorTok{;}
    \VariableTok{osc}\NormalTok{.}\AttributeTok{start}\NormalTok{(}\VariableTok{audioContext}\NormalTok{.}\AttributeTok{currentTime}\NormalTok{)}\OperatorTok{;}
\NormalTok{    is_sound_on }\OperatorTok{=} \KeywordTok{true}\OperatorTok{;}
    \AttributeTok{changeSoundStatus}\NormalTok{(}\StringTok{'True'}\NormalTok{)}\OperatorTok{;}
    \AttributeTok{changeTextWaveform}\NormalTok{(}\StringTok{'Highlight'}\OperatorTok{,} \VariableTok{osc}\NormalTok{.}\AttributeTok{type}\NormalTok{)}\OperatorTok{;}
\OperatorTok{\};}
\KeywordTok{var}\NormalTok{ endSound }\OperatorTok{=} \KeywordTok{function}\NormalTok{() }\OperatorTok{\{}
    \ControlFlowTok{if}\NormalTok{ (osc }\OperatorTok{==} \KeywordTok{undefined}\NormalTok{) }\OperatorTok{\{}\ControlFlowTok{return}\OperatorTok{;\}}
    \VariableTok{osc}\NormalTok{.}\AttributeTok{stop}\NormalTok{(}\VariableTok{audioContext}\NormalTok{.}\AttributeTok{currentTime}\NormalTok{)}\OperatorTok{;}
\NormalTok{    is_sound_on }\OperatorTok{=} \KeywordTok{false}\OperatorTok{;}
    \AttributeTok{changeSoundStatus}\NormalTok{(}\StringTok{'False'}\NormalTok{)}\OperatorTok{;}
    \AttributeTok{changeTextWaveform}\NormalTok{(}\StringTok{'NoHighlight'}\OperatorTok{,} \StringTok{'None'}\NormalTok{)}\OperatorTok{;}
\OperatorTok{\};}
\end{Highlighting}
\end{Shaded}

If we are using this script with the HTML webpage (which we are), then
starting or stopping the sound can change the text in the interface. If,
for some reason, we aren't in the webpage, then that function simply
won't return anything.

\begin{Shaded}
\begin{Highlighting}[]
\KeywordTok{var}\NormalTok{ changeSoundStatus }\OperatorTok{=} \KeywordTok{function}\NormalTok{ (status) }\OperatorTok{\{}
    \KeywordTok{var}\NormalTok{ span }\OperatorTok{=} \VariableTok{document}\NormalTok{.}\AttributeTok{getElementById}\NormalTok{(}\StringTok{'sound_status'}\NormalTok{)}\OperatorTok{;}
    \VariableTok{span}\NormalTok{.}\AttributeTok{innerHTML} \OperatorTok{=}\NormalTok{ status}\OperatorTok{;}
    \VariableTok{span}\NormalTok{.}\AttributeTok{className} \OperatorTok{=}\NormalTok{ status}\OperatorTok{;}
\OperatorTok{\};}
\KeywordTok{var}\NormalTok{ changeTextWaveform }\OperatorTok{=} \KeywordTok{function}\NormalTok{ (status}\OperatorTok{,}\NormalTok{ waveform) }\OperatorTok{\{}
    \VariableTok{document}\NormalTok{.}\AttributeTok{getElementById}\NormalTok{(}\StringTok{'waveform_type'}\NormalTok{).}\AttributeTok{innerHTML} \OperatorTok{=}\NormalTok{ waveform}\OperatorTok{;}
    \VariableTok{document}\NormalTok{.}\AttributeTok{getElementById}\NormalTok{(}\StringTok{'waveform_type'}\NormalTok{).}\AttributeTok{className} \OperatorTok{=}\NormalTok{ status}\OperatorTok{;}
\OperatorTok{\};}
\KeywordTok{var}\NormalTok{ humanHearingCheck }\OperatorTok{=} \KeywordTok{function}\NormalTok{(frequency) }\OperatorTok{\{}
    \ControlFlowTok{return}\OperatorTok{;}
\OperatorTok{\};}
\end{Highlighting}
\end{Shaded}

\end{document}
