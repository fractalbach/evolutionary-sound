\documentclass[]{article}
\usepackage{lmodern}
\usepackage[hyphenbreaks]{breakurl}
\usepackage[hyphens]{url}
\usepackage{fullpage}
\usepackage{amssymb,amsmath}
\usepackage{ifxetex,ifluatex}
\usepackage{fixltx2e} % provides \textsubscript
\ifnum 0\ifxetex 1\fi\ifluatex 1\fi=0 % if pdftex
  \usepackage[T1]{fontenc}
  \usepackage[utf8]{inputenc}
\else % if luatex or xelatex
  \ifxetex
    \usepackage{mathspec}
  \else
    \usepackage{fontspec}
  \fi
  \defaultfontfeatures{Ligatures=TeX,Scale=MatchLowercase}
\fi
% use upquote if available, for straight quotes in verbatim environments
\IfFileExists{upquote.sty}{\usepackage{upquote}}{}
% use microtype if available
\IfFileExists{microtype.sty}{%
\usepackage{microtype}
\UseMicrotypeSet[protrusion]{basicmath} % disable protrusion for tt fonts
}{}
\usepackage[unicode=true]{hyperref}
\hypersetup{
            pdfborder={0 0 0},
            breaklinks=true}
\urlstyle{same}  % don't use monospace font for urls
\usepackage{color}
\usepackage{fancyvrb}
\newcommand{\VerbBar}{|}
\newcommand{\VERB}{\Verb[commandchars=\\\{\}]}
\DefineVerbatimEnvironment{Highlighting}{Verbatim}{commandchars=\\\{\}}
% Add ',fontsize=\small' for more characters per line
\newenvironment{Shaded}{}{}
\newcommand{\KeywordTok}[1]{\textbf{#1}}
\newcommand{\DataTypeTok}[1]{\underline{#1}}
\newcommand{\DecValTok}[1]{#1}
\newcommand{\BaseNTok}[1]{#1}
\newcommand{\FloatTok}[1]{#1}
\newcommand{\ConstantTok}[1]{#1}
\newcommand{\CharTok}[1]{#1}
\newcommand{\SpecialCharTok}[1]{#1}
\newcommand{\StringTok}[1]{#1}
\newcommand{\VerbatimStringTok}[1]{#1}
\newcommand{\SpecialStringTok}[1]{#1}
\newcommand{\ImportTok}[1]{#1}
\newcommand{\CommentTok}[1]{\textit{#1}}
\newcommand{\DocumentationTok}[1]{\textit{#1}}
\newcommand{\AnnotationTok}[1]{\textit{#1}}
\newcommand{\CommentVarTok}[1]{\textit{#1}}
\newcommand{\OtherTok}[1]{#1}
\newcommand{\FunctionTok}[1]{#1}
\newcommand{\VariableTok}[1]{#1}
\newcommand{\ControlFlowTok}[1]{\textbf{#1}}
\newcommand{\OperatorTok}[1]{#1}
\newcommand{\BuiltInTok}[1]{#1}
\newcommand{\ExtensionTok}[1]{#1}
\newcommand{\PreprocessorTok}[1]{\textbf{#1}}
\newcommand{\AttributeTok}[1]{#1}
\newcommand{\RegionMarkerTok}[1]{#1}
\newcommand{\InformationTok}[1]{\textit{#1}}
\newcommand{\WarningTok}[1]{\textit{#1}}
\newcommand{\AlertTok}[1]{\textbf{#1}}
\newcommand{\ErrorTok}[1]{\textbf{#1}}
\newcommand{\NormalTok}[1]{#1}
\IfFileExists{parskip.sty}{%
\usepackage{parskip}
}{% else
\setlength{\parindent}{0pt}
\setlength{\parskip}{6pt plus 2pt minus 1pt}
}
\setlength{\emergencystretch}{3em}  % prevent overfull lines
\providecommand{\tightlist}{%
  \setlength{\itemsep}{0pt}\setlength{\parskip}{0pt}}
\setcounter{secnumdepth}{0}
% Redefines (sub)paragraphs to behave more like sections
\ifx\paragraph\undefined\else
\let\oldparagraph\paragraph
\renewcommand{\paragraph}[1]{\oldparagraph{#1}\mbox{}}
\fi
\ifx\subparagraph\undefined\else
\let\oldsubparagraph\subparagraph
\renewcommand{\subparagraph}[1]{\oldsubparagraph{#1}\mbox{}}
\fi

% set default figure placement to htbp
\makeatletter
\def\fps@figure{htbp}
\makeatother


\date{}


\begin{document}


\section{Sound Check Experiment}\label{sound-check-experiment}

This is a preliminary test for making sound with javascript. Right now,
the code will be tangled together in an attempt to understand how
everything works. The actual program will have much more separation of
parts.

\subsection{Global Constants}\label{global-constants}

The Lists will be constant throughout the whole program. The program
will randomly select an item from one of these lists depending on the
situation, and use that value to affect the sound in some way.

\begin{Shaded}
\begin{Highlighting}[]
\KeywordTok{const}\NormalTok{ LIST_OF_WAVEFORM_TYPES }\OperatorTok{=}\NormalTok{ [}
    \StringTok{'sine'}\OperatorTok{,} \StringTok{'square'}\OperatorTok{,} \StringTok{'sawtooth'}\OperatorTok{,} \StringTok{'triangle'}
\NormalTok{]}\OperatorTok{;}
\end{Highlighting}
\end{Shaded}

\subsection{Global Functions}\label{global-functions}

During the creation of random sounds, randomly generating numbers is a
common occurence. This function generates a random integer between two
integers.

More precisely, let \(f(m,M)\) be a function
\(f:(\mathbb{Z}, \mathbb{Z}) \to \mathbb{Z}\) such that \(f(m,M) = x\)
and \(m \leq x \leq M\)

\begin{Shaded}
\begin{Highlighting}[]
\KeywordTok{function} \AttributeTok{getRandomInt}\NormalTok{(min}\OperatorTok{,}\NormalTok{ max) }\OperatorTok{\{}
\NormalTok{    min }\OperatorTok{=} \VariableTok{Math}\NormalTok{.}\AttributeTok{ceil}\NormalTok{(min)}\OperatorTok{;}
\NormalTok{    max }\OperatorTok{=} \VariableTok{Math}\NormalTok{.}\AttributeTok{floor}\NormalTok{(max) }\OperatorTok{+} \DecValTok{1}\OperatorTok{;}
    \ControlFlowTok{return} \VariableTok{Math}\NormalTok{.}\AttributeTok{floor}\NormalTok{(}\VariableTok{Math}\NormalTok{.}\AttributeTok{random}\NormalTok{() }\OperatorTok{*}\NormalTok{ (max }\OperatorTok{-}\NormalTok{ min)) }\OperatorTok{+}\NormalTok{ min}\OperatorTok{;}
\OperatorTok{\}}
\end{Highlighting}
\end{Shaded}

Since the most common target is a random value from one of the global
Constant Arrays, a useful function will return one of those randomly
selected values. This can be achieved by picking a random KEY in the
array, with the minimum being 0, and the maximum being the length of the
array.

\begin{Shaded}
\begin{Highlighting}[]
\KeywordTok{function} \AttributeTok{getRandomValueFromArray}\NormalTok{(name) }\OperatorTok{\{}
    \ControlFlowTok{return}\NormalTok{ name[ }\AttributeTok{getRandomInt}\NormalTok{(}\DecValTok{0}\OperatorTok{,} \VariableTok{name}\NormalTok{.}\AttributeTok{length} \OperatorTok{-} \DecValTok{1}\NormalTok{) ]}\OperatorTok{;}
\OperatorTok{\}}
\end{Highlighting}
\end{Shaded}

\subsection{Audio Context}\label{audio-context}

Quote from
\href{https://developer.mozilla.org/en-US/docs/Web/API/AudioContext}{Mozilla
Developer - Audio Context}

\begin{quote}
The AudioContext interface represents an audio-processing graph built
from audio modules linked together, each represented by an AudioNode. An
audio context controls both the creation of the nodes it contains and
the execution of the audio processing, or decoding. You need to create
an AudioContext before you do anything else, as everything happens
inside a context.
\end{quote}

\begin{Shaded}
\begin{Highlighting}[]
\KeywordTok{var}\NormalTok{ audioContext }\OperatorTok{=} \KeywordTok{new}\NormalTok{ (}\VariableTok{window}\NormalTok{.}\AttributeTok{AudioContext} \OperatorTok{||} \VariableTok{window}\NormalTok{.}\AttributeTok{webkitAudioContext}\NormalTok{)()}\OperatorTok{;}
\end{Highlighting}
\end{Shaded}

\subsection{Simple Oscillator}\label{simple-oscillator}

For this example, an oscillator node will be created. Given that we have
a list of waveforms, we will choose one of them at random upon creation
of the oscillator.

\begin{Shaded}
\begin{Highlighting}[]
\KeywordTok{var}\NormalTok{ buildRandomOscillator }\OperatorTok{=} \KeywordTok{function}\NormalTok{ () }\OperatorTok{\{}
    \KeywordTok{var}\NormalTok{ osc}\OperatorTok{;}
\NormalTok{    osc }\OperatorTok{=} \VariableTok{audioContext}\NormalTok{.}\AttributeTok{createOscillator}\NormalTok{()}\OperatorTok{;}
    \AttributeTok{RandomizeOscillator}\NormalTok{(osc)}\OperatorTok{;}
    \VariableTok{osc}\NormalTok{.}\AttributeTok{start}\NormalTok{(}\VariableTok{audioContext}\NormalTok{.}\AttributeTok{currentTime}\NormalTok{)}\OperatorTok{;}
    \VariableTok{console}\NormalTok{.}\AttributeTok{log}\NormalTok{(}\StringTok{'New Oscillator Created!}\SpecialCharTok{\textbackslash{}n}\StringTok{'}\OperatorTok{,}\NormalTok{ osc)}\OperatorTok{;}
    \ControlFlowTok{return}\NormalTok{ osc}\OperatorTok{;}
\OperatorTok{\};}  
\end{Highlighting}
\end{Shaded}

RandomizeOscillator will take an existing oscillator and give it random
values for the type and frequency.

\begin{Shaded}
\begin{Highlighting}[]
\KeywordTok{var}\NormalTok{ RandomizeOscillator }\OperatorTok{=} \KeywordTok{function}\NormalTok{ (osc) }\OperatorTok{\{}
    \VariableTok{osc}\NormalTok{.}\AttributeTok{type} \OperatorTok{=} \AttributeTok{getRandomValueFromArray}\NormalTok{(LIST_OF_WAVEFORM_TYPES)}\OperatorTok{;}
    \VariableTok{osc}\NormalTok{.}\VariableTok{frequency}\NormalTok{.}\AttributeTok{value} \OperatorTok{=} \AttributeTok{getRandomInt}\NormalTok{(}\DecValTok{400}\OperatorTok{,}\DecValTok{100}\NormalTok{)}\OperatorTok{;}
    \ControlFlowTok{return}\NormalTok{ osc}\OperatorTok{;}
\OperatorTok{\};}
\end{Highlighting}
\end{Shaded}

\section{Button Events}\label{button-events}

\subsection{Start and Stop the Sound}\label{start-and-stop-the-sound}

For now, we want an oscillator to be defined globally, so that we can
turn it on or off. Before any buttons are pressed, there is no sound. We
just have an empty variable where the oscillator will go when it is
turned on.

\begin{Shaded}
\begin{Highlighting}[]
\KeywordTok{var}\NormalTok{ osc}\OperatorTok{;}
\KeywordTok{var}\NormalTok{ is_sound_on }\OperatorTok{=} \KeywordTok{false}\OperatorTok{;}
\NormalTok{osc }\OperatorTok{=} \AttributeTok{buildRandomOscillator}\NormalTok{()}\OperatorTok{;}
\end{Highlighting}
\end{Shaded}

Starting the sound will \emph{Connect} the oscillator to the speakers,
effectively turning it on for the listener.

\begin{Shaded}
\begin{Highlighting}[]
\KeywordTok{var}\NormalTok{ startSound }\OperatorTok{=} \KeywordTok{function}\NormalTok{() }\OperatorTok{\{}
    \VariableTok{osc}\NormalTok{.}\AttributeTok{connect}\NormalTok{(}\VariableTok{audioContext}\NormalTok{.}\AttributeTok{destination}\NormalTok{)}\OperatorTok{;}
\NormalTok{    is_sound_on }\OperatorTok{=} \KeywordTok{true}\OperatorTok{;}
\OperatorTok{\};}
\end{Highlighting}
\end{Shaded}

Ending the sound will \emph{Disconnect} the oscillator from the
speakers, which will effectively turn the sound off.

\begin{Shaded}
\begin{Highlighting}[]
\KeywordTok{var}\NormalTok{ endSound }\OperatorTok{=} \KeywordTok{function}\NormalTok{() }\OperatorTok{\{}
    \VariableTok{osc}\NormalTok{.}\AttributeTok{disconnect}\NormalTok{(}\VariableTok{audioContext}\NormalTok{.}\AttributeTok{destination}\NormalTok{)}\OperatorTok{;}
\NormalTok{    is_sound_on }\OperatorTok{=} \KeywordTok{false}\OperatorTok{;}
\OperatorTok{\};}
\end{Highlighting}
\end{Shaded}

Toggle sound simply decides if the audio should be turned on or off
based on its current status.

\begin{Shaded}
\begin{Highlighting}[]
\KeywordTok{var}\NormalTok{ toggleSound }\OperatorTok{=} \KeywordTok{function}\NormalTok{ () }\OperatorTok{\{}
    \ControlFlowTok{if}\NormalTok{  (}\OperatorTok{!}\NormalTok{is_sound_on)   }\OperatorTok{\{}\ControlFlowTok{return} \AttributeTok{startSound}\NormalTok{()}\OperatorTok{;\}}
    \ControlFlowTok{if}\NormalTok{  (is_sound_on)    }\OperatorTok{\{}\ControlFlowTok{return} \AttributeTok{endSound}\NormalTok{()}\OperatorTok{;\}}
\OperatorTok{\};}
\end{Highlighting}
\end{Shaded}

\subsection{Changing the Interface}\label{changing-the-interface}

\textbf{updateInterface()} will call of the text-altering functions that
affect what is displayed on the screen. Globalized variables are
converted into strings, and then into HTML.

\textbf{Note:} Functions outside of the interface call upon this one. It
connects the actual sound events to the interface. Everything else, and
how the interface works, is defined further in this section.

\begin{Shaded}
\begin{Highlighting}[]
\KeywordTok{var}\NormalTok{ updateInterface }\OperatorTok{=} \KeywordTok{function}\NormalTok{ () }\OperatorTok{\{}
    \AttributeTok{changeSoundStatus}\NormalTok{(is_sound_on)}\OperatorTok{;}
    \AttributeTok{changeTextWaveform}\NormalTok{(is_sound_on}\OperatorTok{,} \VariableTok{osc}\NormalTok{.}\AttributeTok{type}\NormalTok{)}\OperatorTok{;}
    \AttributeTok{changeTextFrequency}\NormalTok{(is_sound_on}\OperatorTok{,} \VariableTok{osc}\NormalTok{.}\VariableTok{frequency}\NormalTok{.}\AttributeTok{value}\NormalTok{)}\OperatorTok{;}
\OperatorTok{\};}
\end{Highlighting}
\end{Shaded}

Map TRUE and FALSE boolean values into strings

\begin{Shaded}
\begin{Highlighting}[]
\KeywordTok{var}\NormalTok{ boolHighL }\OperatorTok{=} \KeywordTok{function}\NormalTok{ (status) }\OperatorTok{\{}
    \ControlFlowTok{if}\NormalTok{ (status) }\OperatorTok{\{}\ControlFlowTok{return} \StringTok{'Highlight'}\OperatorTok{;\}} 
    \ControlFlowTok{else}        \OperatorTok{\{}\ControlFlowTok{return} \StringTok{'NoHighlight'}\OperatorTok{;\}}
\OperatorTok{\};}
\end{Highlighting}
\end{Shaded}

These functions interact with the interface (which is the browser).

\begin{Shaded}
\begin{Highlighting}[]
\KeywordTok{var}\NormalTok{ changeSoundStatus }\OperatorTok{=} \KeywordTok{function}\NormalTok{ (status) }\OperatorTok{\{}
    \KeywordTok{var}\NormalTok{ span }\OperatorTok{=} \VariableTok{document}\NormalTok{.}\AttributeTok{getElementById}\NormalTok{(}\StringTok{'sound_status'}\NormalTok{)}\OperatorTok{;}
    \VariableTok{span}\NormalTok{.}\AttributeTok{innerHTML} \OperatorTok{=} \VariableTok{status}\NormalTok{.}\AttributeTok{toString}\NormalTok{()}\OperatorTok{;}
    \VariableTok{span}\NormalTok{.}\AttributeTok{className} \OperatorTok{=} \VariableTok{status}\NormalTok{.}\AttributeTok{toString}\NormalTok{()}\OperatorTok{;}
\OperatorTok{\};}
\KeywordTok{var}\NormalTok{ changeTextWaveform }\OperatorTok{=} \KeywordTok{function}\NormalTok{ (status}\OperatorTok{,}\NormalTok{ waveform) }\OperatorTok{\{}
    \VariableTok{document}\NormalTok{.}\AttributeTok{getElementById}\NormalTok{(}\StringTok{'waveform_type'}\NormalTok{).}\AttributeTok{innerHTML} \OperatorTok{=}\NormalTok{ waveform}\OperatorTok{;}
    \VariableTok{document}\NormalTok{.}\AttributeTok{getElementById}\NormalTok{(}\StringTok{'waveform_type'}\NormalTok{).}\AttributeTok{className} \OperatorTok{=} \AttributeTok{boolHighL}\NormalTok{(status)}\OperatorTok{;}
\OperatorTok{\};}
\KeywordTok{var}\NormalTok{ changeTextFrequency }\OperatorTok{=} \KeywordTok{function}\NormalTok{ (status}\OperatorTok{,}\NormalTok{ frequency) }\OperatorTok{\{}
    \VariableTok{document}\NormalTok{.}\AttributeTok{getElementById}\NormalTok{(}\StringTok{'span_frequency'}\NormalTok{).}\AttributeTok{innerHTML} \OperatorTok{=} \VariableTok{frequency}\NormalTok{.}\AttributeTok{toString}\NormalTok{()}\OperatorTok{;}
    \VariableTok{document}\NormalTok{.}\AttributeTok{getElementById}\NormalTok{(}\StringTok{'span_frequency'}\NormalTok{).}\AttributeTok{className} \OperatorTok{=} \AttributeTok{boolHighL}\NormalTok{(status)}\OperatorTok{;} 
\OperatorTok{\};}
\KeywordTok{var}\NormalTok{ humanHearingCheck }\OperatorTok{=} \KeywordTok{function}\NormalTok{(frequency) }\OperatorTok{\{}
    \ControlFlowTok{return}\OperatorTok{;}
\OperatorTok{\};}
\end{Highlighting}
\end{Shaded}

\end{document}
