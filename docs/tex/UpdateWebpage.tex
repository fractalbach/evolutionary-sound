\documentclass[]{article}
\usepackage{lmodern}
\usepackage[hyphenbreaks]{breakurl}
\usepackage[hyphens]{url}
\usepackage{fullpage}
\usepackage{amssymb,amsmath}
\usepackage{ifxetex,ifluatex}
\usepackage{fixltx2e} % provides \textsubscript
\ifnum 0\ifxetex 1\fi\ifluatex 1\fi=0 % if pdftex
  \usepackage[T1]{fontenc}
  \usepackage[utf8]{inputenc}
\else % if luatex or xelatex
  \ifxetex
    \usepackage{mathspec}
  \else
    \usepackage{fontspec}
  \fi
  \defaultfontfeatures{Ligatures=TeX,Scale=MatchLowercase}
\fi
% use upquote if available, for straight quotes in verbatim environments
\IfFileExists{upquote.sty}{\usepackage{upquote}}{}
% use microtype if available
\IfFileExists{microtype.sty}{%
\usepackage{microtype}
\UseMicrotypeSet[protrusion]{basicmath} % disable protrusion for tt fonts
}{}
\usepackage[unicode=true]{hyperref}
\hypersetup{
            pdfborder={0 0 0},
            breaklinks=true}
\urlstyle{same}  % don't use monospace font for urls
\usepackage{color}
\usepackage{fancyvrb}
\newcommand{\VerbBar}{|}
\newcommand{\VERB}{\Verb[commandchars=\\\{\}]}
\DefineVerbatimEnvironment{Highlighting}{Verbatim}{commandchars=\\\{\}}
% Add ',fontsize=\small' for more characters per line
\newenvironment{Shaded}{}{}
\newcommand{\KeywordTok}[1]{\textcolor[rgb]{0.00,0.44,0.13}{\textbf{#1}}}
\newcommand{\DataTypeTok}[1]{\textcolor[rgb]{0.56,0.13,0.00}{#1}}
\newcommand{\DecValTok}[1]{\textcolor[rgb]{0.25,0.63,0.44}{#1}}
\newcommand{\BaseNTok}[1]{\textcolor[rgb]{0.25,0.63,0.44}{#1}}
\newcommand{\FloatTok}[1]{\textcolor[rgb]{0.25,0.63,0.44}{#1}}
\newcommand{\ConstantTok}[1]{\textcolor[rgb]{0.53,0.00,0.00}{#1}}
\newcommand{\CharTok}[1]{\textcolor[rgb]{0.25,0.44,0.63}{#1}}
\newcommand{\SpecialCharTok}[1]{\textcolor[rgb]{0.25,0.44,0.63}{#1}}
\newcommand{\StringTok}[1]{\textcolor[rgb]{0.25,0.44,0.63}{#1}}
\newcommand{\VerbatimStringTok}[1]{\textcolor[rgb]{0.25,0.44,0.63}{#1}}
\newcommand{\SpecialStringTok}[1]{\textcolor[rgb]{0.73,0.40,0.53}{#1}}
\newcommand{\ImportTok}[1]{#1}
\newcommand{\CommentTok}[1]{\textcolor[rgb]{0.38,0.63,0.69}{\textit{#1}}}
\newcommand{\DocumentationTok}[1]{\textcolor[rgb]{0.73,0.13,0.13}{\textit{#1}}}
\newcommand{\AnnotationTok}[1]{\textcolor[rgb]{0.38,0.63,0.69}{\textbf{\textit{#1}}}}
\newcommand{\CommentVarTok}[1]{\textcolor[rgb]{0.38,0.63,0.69}{\textbf{\textit{#1}}}}
\newcommand{\OtherTok}[1]{\textcolor[rgb]{0.00,0.44,0.13}{#1}}
\newcommand{\FunctionTok}[1]{\textcolor[rgb]{0.02,0.16,0.49}{#1}}
\newcommand{\VariableTok}[1]{\textcolor[rgb]{0.10,0.09,0.49}{#1}}
\newcommand{\ControlFlowTok}[1]{\textcolor[rgb]{0.00,0.44,0.13}{\textbf{#1}}}
\newcommand{\OperatorTok}[1]{\textcolor[rgb]{0.40,0.40,0.40}{#1}}
\newcommand{\BuiltInTok}[1]{#1}
\newcommand{\ExtensionTok}[1]{#1}
\newcommand{\PreprocessorTok}[1]{\textcolor[rgb]{0.74,0.48,0.00}{#1}}
\newcommand{\AttributeTok}[1]{\textcolor[rgb]{0.49,0.56,0.16}{#1}}
\newcommand{\RegionMarkerTok}[1]{#1}
\newcommand{\InformationTok}[1]{\textcolor[rgb]{0.38,0.63,0.69}{\textbf{\textit{#1}}}}
\newcommand{\WarningTok}[1]{\textcolor[rgb]{0.38,0.63,0.69}{\textbf{\textit{#1}}}}
\newcommand{\AlertTok}[1]{\textcolor[rgb]{1.00,0.00,0.00}{\textbf{#1}}}
\newcommand{\ErrorTok}[1]{\textcolor[rgb]{1.00,0.00,0.00}{\textbf{#1}}}
\newcommand{\NormalTok}[1]{#1}
\IfFileExists{parskip.sty}{%
\usepackage{parskip}
}{% else
\setlength{\parindent}{0pt}
\setlength{\parskip}{6pt plus 2pt minus 1pt}
}
\setlength{\emergencystretch}{3em}  % prevent overfull lines
\providecommand{\tightlist}{%
  \setlength{\itemsep}{0pt}\setlength{\parskip}{0pt}}
\setcounter{secnumdepth}{0}
% Redefines (sub)paragraphs to behave more like sections
\ifx\paragraph\undefined\else
\let\oldparagraph\paragraph
\renewcommand{\paragraph}[1]{\oldparagraph{#1}\mbox{}}
\fi
\ifx\subparagraph\undefined\else
\let\oldsubparagraph\subparagraph
\renewcommand{\subparagraph}[1]{\oldsubparagraph{#1}\mbox{}}
\fi

% set default figure placement to htbp
\makeatletter
\def\fps@figure{htbp}
\makeatother


\date{}


\begin{document}


\section{Updating the Webpage}\label{updating-the-webpage}

Updating the webpage is built on a layer all of the functionality of the
program. If this javascript file were not included, you could
theoretically use the entire program with the command line. However, you
must have the functionality in order to use this javascript.

Several Types of Events are handled in this file:

\begin{itemize}
\tightlist
\item
  Functions that can be used by buttons and other interactive things.
\item
  changing actual HTML and Text on the webpage.
\item
  changing attributes to various DOM elements.
\item
  drawing onto a canvas.
\end{itemize}

\subsection{Updating Interfaces}\label{updating-interfaces}

Currently, \textbf{updateInterface()} is the only function here. This
clumps together the various individual components that are usually
updated together. Button clicks will most probably call this function.

\begin{Shaded}
\begin{Highlighting}[]
\KeywordTok{var}\NormalTok{ updateInterface }\OperatorTok{=} \KeywordTok{function}\NormalTok{ () }\OperatorTok{\{}
    \AttributeTok{changeSoundStatus}\NormalTok{(is_sound_on)}\OperatorTok{;}
    \AttributeTok{changeTextWaveform}\NormalTok{(is_sound_on}\OperatorTok{,} \VariableTok{osc}\NormalTok{.}\AttributeTok{type}\NormalTok{)}\OperatorTok{;}
    \AttributeTok{changeTextFrequency}\NormalTok{(is_sound_on}\OperatorTok{,} \VariableTok{osc}\NormalTok{.}\VariableTok{frequency}\NormalTok{.}\AttributeTok{value}\NormalTok{)}\OperatorTok{;}
\OperatorTok{\};}
\end{Highlighting}
\end{Shaded}

\subsection{Changing Text}\label{changing-text}

Here is where specific elements within the HTML document are targeted
and changed. The functions here are very dependent on the structure of
the HTML and the unique IDs they have been given.

\textbf{changeSoundStatus(boolean)} tells you if the sound is ON or OFF.

\begin{Shaded}
\begin{Highlighting}[]
\KeywordTok{var}\NormalTok{ changeSoundStatus }\OperatorTok{=} \KeywordTok{function}\NormalTok{ (status) }\OperatorTok{\{}
    \KeywordTok{var}\NormalTok{ span }\OperatorTok{=} \VariableTok{document}\NormalTok{.}\AttributeTok{getElementById}\NormalTok{(}\StringTok{'sound_status'}\NormalTok{)}\OperatorTok{;}
    \VariableTok{span}\NormalTok{.}\AttributeTok{innerHTML} \OperatorTok{=} \VariableTok{status}\NormalTok{.}\AttributeTok{toString}\NormalTok{()}\OperatorTok{;}
    \VariableTok{span}\NormalTok{.}\AttributeTok{className} \OperatorTok{=} \VariableTok{status}\NormalTok{.}\AttributeTok{toString}\NormalTok{()}\OperatorTok{;}
\OperatorTok{\};}
\end{Highlighting}
\end{Shaded}

\textbf{changeTextWaveform(boolean, string)} displays the type of
waveform being used by given oscillator. The string will most likely be
derived from oscillator.type

\begin{Shaded}
\begin{Highlighting}[]
\KeywordTok{var}\NormalTok{ changeTextWaveform }\OperatorTok{=} \KeywordTok{function}\NormalTok{ (status}\OperatorTok{,}\NormalTok{ waveform) }\OperatorTok{\{}
    \VariableTok{document}\NormalTok{.}\AttributeTok{getElementById}\NormalTok{(}\StringTok{'waveform_type'}\NormalTok{).}\AttributeTok{innerHTML} \OperatorTok{=}\NormalTok{ waveform}\OperatorTok{;}
    \VariableTok{document}\NormalTok{.}\AttributeTok{getElementById}\NormalTok{(}\StringTok{'waveform_type'}\NormalTok{).}\AttributeTok{className} \OperatorTok{=} \AttributeTok{boolHighL}\NormalTok{(status)}\OperatorTok{;}
\OperatorTok{\};}
\end{Highlighting}
\end{Shaded}

\textbf{changeTextFrequency(boolean, integer)} displays the oscillator's
frequency. However, this function would accept and display any integer
given.

\begin{Shaded}
\begin{Highlighting}[]
\KeywordTok{var}\NormalTok{ changeTextFrequency }\OperatorTok{=} \KeywordTok{function}\NormalTok{ (status}\OperatorTok{,}\NormalTok{ frequency) }\OperatorTok{\{}
    \VariableTok{document}\NormalTok{.}\AttributeTok{getElementById}\NormalTok{(}\StringTok{'span_frequency'}\NormalTok{).}\AttributeTok{innerHTML} \OperatorTok{=} \VariableTok{frequency}\NormalTok{.}\AttributeTok{toString}\NormalTok{()}\OperatorTok{;}
    \VariableTok{document}\NormalTok{.}\AttributeTok{getElementById}\NormalTok{(}\StringTok{'span_frequency'}\NormalTok{).}\AttributeTok{className} \OperatorTok{=} \AttributeTok{boolHighL}\NormalTok{(status)}\OperatorTok{;} 
\OperatorTok{\};}
\end{Highlighting}
\end{Shaded}

\subsection{Supporting Functions}\label{supporting-functions}

Supporting Functions map values that are saved in objects, and converts
them into a printable form (like a string).

For example, this function takes a boolean and converts it into
``highlight'' or ``no highlight''. These are the names of classes in the
CSS that style boxes different colors.

\begin{Shaded}
\begin{Highlighting}[]
\KeywordTok{var}\NormalTok{ boolHighL }\OperatorTok{=} \KeywordTok{function}\NormalTok{ (status) }\OperatorTok{\{}
    \ControlFlowTok{if}\NormalTok{ (status) }\OperatorTok{\{}\ControlFlowTok{return} \StringTok{'Highlight'}\OperatorTok{;\}} 
    \ControlFlowTok{else}        \OperatorTok{\{}\ControlFlowTok{return} \StringTok{'NoHighlight'}\OperatorTok{;\}}
\OperatorTok{\};}
\end{Highlighting}
\end{Shaded}

\end{document}
